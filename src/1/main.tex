\documentclass{article}

% Language setting
% Replace `english' with e.g. `spanish' to change the document language
\usepackage[english]{babel}

% Set page size and margins
% Replace `letterpaper' with `a4paper' for UK/EU standard size
\usepackage[letterpaper,top=2cm,bottom=2cm,left=3cm,right=3cm,marginparwidth=1.75cm]{geometry}

% Useful packages
\usepackage{amsmath}
\usepackage{graphicx}
\usepackage[colorlinks=true, allcolors=blue]{hyperref}
%ini default overleaf, lupa saya hapus
\title{Esai diri THT2 RSC 2026}
\author{Richard Samuel Simanullang}

\begin{document}
\maketitle

\quad Saya mulai tertarik dengan sains dan teknologi semenjak menginjak Sekolah Dasar. Acara TV dan berita tentang penerbangan serta eksplorasi ruang angkasa membuat saya semakin tertarik dengan ilmu penerbangan dan antariksa. Saya juga sering membaca buku tentang pesawat dan perkembangannya namun karena masih kecil saya tidak terlalu paham dengan hal teknisnya. Selain itu, saya juga tertarik dengan teknologi apalagi saya tumbuh besar pada masa perkembangan pesat gadget modern. Smartphone layar sentuh, sosial media, serta streaming film merupakan terobosan-terobosan baru yang membuat saya tertarik dengan perkembangan teknologi saat itu.

Kedua hobi tersebut tetap saya miliki hingga menginjak tahap SMA, saya cukup senang dengan mata pelajaran Fisika. Kemudian karena ketertarikan saya di bidang astronomi, saya berhasil menjadi finalis Olimpiade Nasional di bidang Astronomi. Di olimpiade astronomi saya cenderung lebih suka hal terkait mekanika benda langit, peluncuran pesawat ruang angkasa, dan bidang astrofisika lainnya. Pada saat memilih jurusan kuliah, saya sempat mempertimbangkan untuk mengambil Teknik Dirgantara di ITB, namun karena ketertarikan saya terhadap teknologi disertai berbagai alasan lain membuat saya lebih memilih Teknik Informatika ITB.

Selama menjalani semester pertama ITB, saya memilih untuk fokus pada pengembangan skill programming seperti mengikuti UKM yang membuat game. Namun setelah beberapa saat, saya melihat ada teman satu jurusan saya mengikuti UKM Aksantara yang terkenal akan pengembangan UAV-nya. Saya sempat heran mengapa teman saya yang jurusannya Teknik Informatika mengikuti UKM yang terkenal dipenuhi oleh jurusan lain seperti Teknik Dirgantara dan Teknik Mesin. Akhirnya saya tahu bahwa di Aksantara ada departemen RSC yang membutuhkan ilmu keinformatikaan dalam pengerjaannya. Karena itulah saya memutuskan untuk mendaftar ke Aksantara ITB 2026. Setelah mengikuti Day 1 Aksantara, saya baru mengetahui bahwa RSC di aksantara juga terbagi atas dua, yaitu ConCept dan GCS. Sebenarnya saya sangat tertarik dengan kedua jurusan tersebut. Namun saya lebih tertarik dengan Computer Vision dan pemrograman low level yang kemungkinan besar akan saya temukan di jurusan ConCept, karenanya saya memilih jurusan ini. 

Jika saya berhasil menjadi bagian Aksantara ITB 2026 maka saya akan berusaha mengerahkan seluruh kemampuan untuk membantu tim dimana saya ditempatkan untuk meraih juara perlombaan. Saya termotivasi khususnya untuk menjadi bagian dari VTOL ataupun LELA dalam mengikuti KRTI kedepannya. 


\end{document}